% GeotechDoc.tex V1.0, 21 January 2015

\documentclass[times]{GeotechAuth}

\usepackage{moreverb}

\usepackage[colorlinks,bookmarks=false,citecolor=red,urlcolor=red]{hyperref}

\newcommand\BibTeX{{\rmfamily B\kern-.05em \textsc{i\kern-.025em b}\kern-.08em
T\kern-.1667em\lower.7ex\hbox{E}\kern-.125emX}}

\def\volumeyear{2015}

\begin{document}

\runningheads{A demonstration of the {\itshape\journalabb} class file}{A.~N.~Other}

\title{A demonstration of the \LaTeXe\ class file for
\itshape{\journalname}}

\author{A.~N.~Other\addressnum{1}}

\address{\addressnum{1}Ice Publishing, One Great George Street, Westminster, London, SW1P 3AA, UK}

\begin{abstract}
This paper describes the use of the \LaTeXe\
\textsf{\journalclass} class file for setting papers for
\emph{\journalnamelc}.
\end{abstract}

\keywords{class file; \LaTeXe; \emph{\journalabb}}

\maketitle

\section{Introduction}
Many authors submitting to research journals use \LaTeXe\ to
prepare their papers. This paper describes the
\textsf{\journalclass} class file which can be used to convert
articles produced with other \LaTeXe\ class files into the correct
form for publication in \emph{\journalnamelc}.

The \textsf{\journalclass} class file preserves much of the
standard \LaTeXe\ interface so that any document which was
produced using the standard \LaTeXe\ \textsf{article} style can
easily be converted to work with the \textsf{\journalclassshort}
style. However, the width of text and typesize will vary from that
of \textsf{article.cls}; therefore, \emph{line breaks will change}
and it is likely that displayed mathematics and tabular material
will need re-setting.

In the following sections we describe how to lay out your code to
use \textsf{\journalclass} to reproduce the typographical look of
\emph{\journalnamelc}. However, this paper is not a guide to
using \LaTeXe\ and we would refer you to any of the many books
available (see, for example, \cite{R1}, \cite{R2} and \cite{R3}).

\subsection{Important note}
You will find links to the Author Guidelines and other resources to help you prepare your paper for publication at:\\
\url{http://www.icevirtuallibrary.com/authors/publish?contentType=journals}

\section{The three golden rules}
Before we proceed, we would like to stress \emph{three golden
rules} that need to be followed to enable the most efficient use
of your code at the typesetting stage:
\begin{enumerate}
\item[(i)] keep your own macros to an absolute minimum;

\item[(ii)] as \TeX\ is designed to make sensible spacing
decisions by itself, do \emph{not} use explicit horizontal or
vertical spacing commands, except in a few accepted (mostly
mathematical) situations, such as \verb"\," before a
differential~d, or \verb"\quad" to separate an equation from its
qualifier;

\item[(iii)] follow the \emph{\journalnamelc} reference style.
\end{enumerate}

\section{Getting started} The \textsf{\journalclassshort} class file should run
on any standard \LaTeXe\ installation. If any of the fonts, style
files or packages it requires are missing from your installation,
they can be found on the \emph{\TeX\ Collection} DVDs or from
CTAN.

\emph{\journalnamelc} is published using Times fonts and this is
achieved by using the \verb"times"
option as\\
\verb"\documentclass[times]{GeotechAuth}".

\noindent If for any reason you have a problem using Times you can
easily resort to Computer Modern fonts by removing the
\verb"times" option.

\begin{figure*}
\setlength{\fboxsep}{0pt}%
\setlength{\fboxrule}{0pt}%
\begin{center}
\begin{boxedverbatim}

\documentclass[times]{GeotechAuth}
%\documentclass[times,doublespace]{GeotechAuth}%For submission

\begin{document}

\runningheads{<Short title>}{<Initials and Surnames>}

\title{<Your title>}

\author{<A. Author\addressnum{1}, S. Else\addressnum{2} \authorand
P. Another\addressnum{1}>}

\address{<\addressnum{1}First author's full postal address
(in this example it is the same as the third author)\\
\addressnum{2}Second author's full postal address>}

\begin{abstract}
<Text>
\end{abstract}

\keywords{<List keywords>}

\maketitle

\section{Introduction}
.
.
.
\end{boxedverbatim}
\end{center}
\caption{Example header text\label{F1}}
\end{figure*}

\section{The article header information}
The heading for any file using \textsf{\journalclass} is shown in
Figure~\ref{F1}.

\subsection{Remarks}
\begin{enumerate}
\item[(i)] In \verb"\runningheads" use `\emph{et~al.}' if there
are three or more authors.

\item[(ii)] For multiple author papers please note the use of \verb"\addressnum" to
link names and addresses.

\item[(iii)] For submitting a double-spaced manuscript, add
\verb"doublespace" as an option to the documentclass line.

\item[(iv)] The abstract should be capable of standing by itself,
in the absence of the body of the article and of the bibliography.
Therefore, it must not contain any reference citations.

\item[(v)] Supply a maximum of six keywords from the G\'eotechnique list:\\ \url{http://www.icevirtuallibrary.com/upload/geotechniquekeywords.pdf}.
Keywords are separated by semicolons.

\end{enumerate}

\section{The body of the article}

\subsection{Mathematics} \textsf{\journalclass} makes the full
functionality of \AmS\/\TeX\ available. We encourage the use of
the \verb"align", \verb"gather" and \verb"multline" environments
for displayed mathematics.

\subsection{Figures and tables} \textsf{\journalclass} includes the
\textsf{graphicx} package for handling figures.

Figures are called in as follows:
\begin{verbatim}
\begin{figure}
\centering
\includegraphics{<figure name>}
\caption{<Figure caption>}
\end{figure}
\end{verbatim}

For further details on how to size figures, etc., with the
\textsf{graphicx} package see, for example, \cite{R1}
or \cite{R3}.

The standard coding for a table is shown in Figure~\ref{F2}. Please note that \textsf{\journalclass} includes the
\textsf{tabls} package to help improve table spacing.

\begin{figure}
\setlength{\fboxsep}{0pt}%
\setlength{\fboxrule}{0pt}%
\begin{center}
\begin{boxedverbatim}
\begin{table}
\caption{<Table caption>}
\small
\centering
\begin{tabular}{<table alignment>}
   %with "|" between columns
\hline
<column headings>\\
\hline
<table entries
(separated by & as usual)>\\
<table entries>\\
.
.
.\\
\hline
\end{tabular}
\end{table}
\end{boxedverbatim}
\end{center}
\caption{Example table layout\label{F2}}
\end{figure}

\subsection{Cross-referencing}
The use of the \LaTeX\ cross-reference system
for figures, tables, equations, etc., is encouraged
(using \verb"\ref{<name>}" and \verb"\label{<name>}").

\subsection{Bibliography}
Please note that the file \textsf{Geotech.bst} is available from
the same download page for those authors using \BibTeX.

Otherwise, the normal commands for producing the reference list
are:
\begin{verbatim}
\begin{thebibliography}{99}
\item <Reference details>
.
.
.
\item <Reference details>
\end{thebibliography}
\end{verbatim}

\subsection{Double spacing}
If you need to double space your document for submission please
use the \verb+doublespace+ option as shown in the sample layout in
Figure~\ref{F1}.

\section{Copyright statement}
Please  be  aware that the use of  this \LaTeXe\ class file is
governed by the following conditions.

\subsection{Copyright}
Copyright \copyright\ \volumeyear\ Ice Publishing, One Great George Street, Westminster, London, SW1P~3AA, UK. All
rights reserved.

\subsection{Rules of use}
This class file is made available for use by authors who wish to
prepare an article for publication in \emph{\journalnamelc}
published by Ice Publishing. The user may not exploit any
part of the class file commercially.

This class file is provided on an \emph{as is}  basis, without
warranties of any kind, either express or implied, including but
not limited to warranties of title, or implied  warranties of
merchantablility or fitness for a particular purpose. There will
be no duty on the author[s] of the software or Ice Publishing
to correct any errors or defects in the software. Any
statutory  rights you may have remain unaffected by your
acceptance of these rules of use.

\section{Acknowledgements}
This class file was developed by Sunrise Setting Ltd,
Paignton, Devon, UK. Website:\\
\url{http://www.sunrise-setting.co.uk}

\begin{thebibliography}{99}
\bibitem[Kopka \& Daly(2003)]{R1}
{Kopka,~H. \& Daly,~P.W.} (2003). \emph{A guide to \LaTeX}, 4th~edn.
Addison-Wesley.

\bibitem[Lamport(1994)]{R2}
{Lamport,~L.} (1994). \emph{\LaTeX: a document preparation system},
2nd~edn. Addison-Wesley.

\bibitem[Mittelbach \& Goossens(2004)]{R3}
{Mittelbach,~F. \& Goossens,~M.} (2004). \emph{The \LaTeX\ companion},
2nd~edn. Addison-Wesley.

\end{thebibliography}

\end{document}
